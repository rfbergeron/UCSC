\documentclass{article}

% If you're new to LaTeX, here's some short tutorials:
% https://www.overleaf.com/learn/latex/Learn_LaTeX_in_30_minutes
% https://en.wikibooks.org/wiki/LaTeX/Basics

% Formatting
\usepackage[utf8]{inputenc}
\usepackage[margin=1in]{geometry}
\usepackage[titletoc,title]{appendix}

% Math
% https://www.overleaf.com/learn/latex/Mathematical_expressions
% https://en.wikibooks.org/wiki/LaTeX/Mathematics
\usepackage{amsmath,amsfonts,amssymb,mathtools}

% Images
% https://www.overleaf.com/learn/latex/Inserting_Images
% https://en.wikibooks.org/wiki/LaTeX/Floats,_Figures_and_Captions
\usepackage{graphicx,float}

% Tables
% https://www.overleaf.com/learn/latex/Tables
% https://en.wikibooks.org/wiki/LaTeX/Tables

% Algorithms
% https://www.overleaf.com/learn/latex/algorithms
% https://en.wikibooks.org/wiki/LaTeX/Algorithms
\usepackage[ruled,vlined]{algorithm2e}
\usepackage{algorithmic}

% Code syntax highlighting
% pdflatex requires a flag to compile with this package and it doesn't
% seem to do anything useful
% https://www.overleaf.com/learn/latex/Code_Highlighting_with_minted
%\usepackage{minted}
%\usemintedstyle{borland}

% References
% https://www.overleaf.com/learn/latex/Bibliography_management_in_LaTeX
% https://en.wikibooks.org/wiki/LaTeX/Bibliography_Management
\usepackage{biblatex}
\addbibresource{references.bib}

% Title content
\title{AM214 Homework 2, Numerical Part}
\author{Robert Bergeron}
\date{October 28, 2020}

\begin{document}

\maketitle

% Question 2
\section*{Question 2}

This equation is a subcritical pitchfork. In this case, at \(r = -1\) there
is only one, stable fixed point at \(x_* = 0\), so the tendency of \(x\) to
approach \(0\) is not going to change at time approaches infinity.

Linearizing \(\frac{dx}{dt} = rx + x^3 - x^5\) about \(x_* = 0, r = -1\), we get
\[f(x) = (x - x_*)\frac{\partial f}{\partial x} + (r - r_*)\frac{\partial f}{\partial r}
= (x - x_*)(-1) + (r - r_*)(0) = -x\]

Then, we solve the differential equation \(\frac{dx}{dt} = -x\) as follows:
\[\frac{dx}{x} = -dt\]
\[\int \frac{dx}{x} = - \int dt\]
\[\ln x = -t + C\]
\[x = e^{-t + C} = De^{-t}\]

Since our numerical solution had inital conditions \(x_* = 0.01, t = 0\), we can
substitute those here to solve for \(D\), finding that \(D = 0.01\).

% Question 3
\section*{Question 3}
In light of what we discussed in class, the function \(x_*(r_{increasing})\)
demonstrates convergence jumping between stable branches as a result of small
disturbances in a function that has the property of hysterisis.

On the interval \(-\infty < r < -\frac{1}{4}\), \(x_* = 0\) is the only fixed point
that exists, \(x_*(r)\) converges to zero and our initial condition(s)
\(x = 0.01 + x_{prev}\) will remain close to \(0.01\).

Once \(r\) has reached \(-\frac{1}{4}\), we still find that the function converges
to \(0\) since our initial condition is closer to it than any other stable branch.

However, as \(r\) passes \(0\), the fixed point at \(x_* = 0\) becomes unstable and
the convergence point rapidly shifts to the positive
stable branch of the bifurcation, since our initial condition is being displaced in
the positive direction. As \(r\) approaches \(\infty\), \(x_*(r)\) continues along
the positive stable branch of the bifurcation.

\pagebreak

% Question 4
\section*{Question 4}
Taken together, the functions \(x_*(r_{increasing})\), \(x_*(r_{decreasing})\),
and \(x_*(r_{numerical})\) demonstrate the effects of small perturbations of
the initial condition and \(r\) in a function that exhibits hysterisis.

\(x_*(r_{inc})\) and \(x_*(r_{dec})\) both have initial conditions starting at
\(x(0) = 0\), but because the bifurcation has different stable branches that vary
with \(r\), experimental results obtained by decreasing \(r\) near the bifurcation
point and tweaking \(x_*\) will be different from those yielded by increasing \(r\)
near the bifurcation point.

Like \(x_*(r_{inc})\), \(x_*(r_{dec})\) jumps from the stable branch on
\(r = x_*^4 - x_*^2\) to \(x_* = 0\) once \(r < -\frac{1}{4}\). However, there
is one difference, in that the fixed points for branch \(r = x_*^4 - x_*^2\)
\emph{disappear} when \(r < -\frac{1}{4}\), instead of merely becoming unstable,
like the fixed point \(x_* = 0\) at \(r = 0\).

\end{document}
